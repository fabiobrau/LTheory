\documentclass[a4paper,11pt]{article}

\usepackage{amsthm}
\usepackage{amsmath}
\usepackage{amssymb}
\usepackage{amsfonts}
\usepackage{comment}
\usepackage{cleveref}
\usepackage[
backend=biber,
style=alphabetic,
sorting=ynt
]{biblatex}
%% Environments
\newtheorem{thm}{Theorem}
\theoremstyle{definition}
\newtheorem{defn}{Definition}
\author{Fabio Brau}
%\institution{University Sant'Anna}
\title{Notes on Learning Theory}
\addbibresource{biblio.bib}

\begin{document}
\maketitle
\tableofcontents
\section{Basic Definitions}
\begin{defn}
  Given a set $P$ of $t$ variables $P=\{p_1,\dots,p_t\}$, a \textit{vector} 
  is an assignment of the variables to the values $\{0,1,\star\}$. The
  symbol $\star$ denotes that the variable is undetermined. Formally the set 
  of all the vectors is $V=\{0,1,\star\}^P$. A particular sub-class of vector 
  is composed by the \textit{total vectors}, indicated by $V_T$, for which every variable is
  determined. Formally $V_T = \{0,1\}^P$.
\end{defn}

Observe that $V$ contains $3^t$ vectors, and $V_T$ contains $2^t$ vectors.

\begin{defn}[Boolean Functions and Concepts]
  A \textit{boolean function} is any function defined over the total vectors
  with output in $\{0,1\}$. Formally, the set $\mathcal{F}$ of the boolean
  functions is $\mathcal{F}=\{0,1\}^{V_T}$. A \textit{concept} is a function
  defined over the whole vectors, $\mathcal{C}=\{0,1\}^{V}$.
\end{defn}

\begin{defn}[Coerence]
  Two vectors $v,w\in V$ are coherent if they take the same values on the
  common determined variables. Let 
  $A = (v^{-1}(\star)\cap w^{-1}(\star))^c$, the two vectors are
  coherent if $v_{\restriction A} = w_{\restriction A}$.
\end{defn}

\begin{defn}[Extension]
  Let $F$ be a boolean function. Let $v\in V$ a vector, and let $T(v)=\left\{
  w\in V_T\,:\, v,w \, \mbox{coherent} \right\}$ be the set of the total vectors
  coherent with $v$. We say that $F$ can be
  extended up to $v$ if and only if takes the same values on all the coherent
  vectors. In formula, 
  \begin{equation}
    F(v) = 1 \iff\forall w\in T(v), F(w)=1
    \label{extension}
  \end{equation}.
\end{defn}

\subsection{Learning Protocols}
\begin{defn}[Example-Learning]
  Let $F$ be a concept. An \textit{example learning} is a random variable, $\mathcal{E}_F$, 
  that returns vector such that $F(v)=1$ accordingly to some discrete
  probability distribution $D$ over $F^{-1}$. In formula, $\mathcal{E} \in F^{-1}(V)$
  random variable such that $\mathbb{P}(\mathcal{E}_F=v)=D(v)$.
\end{defn}

\begin{defn}[Oracle]
  The \textit{oracle}, given a concept $F$ and a vector $v$, returns the value
  of $F(v)$. Formally, the oracle is the map $\mathcal{O}$ defined as follows
  \begin{equation}
    \begin{aligned}
      \mathcal{O}\,:\, &\mathcal{C}\times V \to \{0,1\}\\
      & \left( F,v \right) \mapsto F(v)
    \end{aligned}
    \label{oracle}
  \end{equation}
\end{defn}
\end{document}
